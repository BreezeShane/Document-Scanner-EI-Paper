\documentclass[10pt, conference, compsocconf]{IEEEtran}
\usepackage{cite}
\usepackage{subfigure}
\ifCLASSINFOpdf
\usepackage[pdftex]{graphicx}
\graphicspath{{pdf/}{figures/}}
% and their extensions so you won’t have to specify these with
% every instance of \includegraphics
\DeclareGraphicsExtensions{.pdf,.jpeg,.png}
\else
% or other class option (dvipsone, dvipdf, if not using dvips). graphicx
% will default to the driver specified in the system graphics.cfg if no
% driver is specified.
\usepackage[dvips]{graphicx}
% declare the path(s) where your graphic files are
\graphicspath{{…/eps/}}
% and their extensions so you won’t have to specify fthese with
% every instance of \includegraphics
\DeclareGraphicsExtensions{.eps}
\fi

\usepackage{array}
\usepackage{multirow}
\usepackage{longtable}
\usepackage{rotating}
\usepackage[cmex10]{amsmath}
\usepackage{algorithmic}
\usepackage{array}
\usepackage{mdwmath}
\usepackage{mdwtab}
\usepackage{stfloats}
\usepackage{url}
\usepackage[colorlinks,
linkcolor=blue,
anchorcolor=blue,
citecolor=blue]{hyperref}
\hyphenation{op-tical net-works semi-conduc-tor}


\begin{document}
	%
	% paper title
	% can use linebreaks \\ within to get better formatting as desired
	\title{Document Scanner Based on PaddlePaddle}
	
	\author{\IEEEauthorblockN{Chang Lu$^1$\IEEEauthorrefmark{0},
			Xiaochun Lei$^{* 1,2}$\IEEEauthorrefmark{0},
			Junlin Xie$^1$\IEEEauthorrefmark{0}, 
			Xiaolong Wang$^1$\IEEEauthorrefmark{0} and
			XiangBoge Mu$^1$\IEEEauthorrefmark{0}}
		\IEEEauthorblockA{\IEEEauthorrefmark{0}1.School of computer and information security,\\
			Guilin University of Electronic Technology,
			Guilin 541004, China;}
		\IEEEauthorblockA{\IEEEauthorrefmark{0}2.Guangxi Key Laboratory of Image and Graphic Intelligent Processing, \\Guilin 541004, China;}
		\IEEEauthorblockA{\IEEEauthorrefmark{1}Corresponding author's e-mail: glleixiaochun@qq.com}}
	% use for sepcial paper notices
	%\IEEEspecialpapernotice{(Invited Paper)}
	% make the title area
	\maketitle
	\begin{abstract}
		
	Document scanning, a type of Optical Character Recognition Technology, aims at trasferring the documents in captured photographs into scanned document files. 
	And with the need of convience to gain contents in documents increasing, it is becoming applied to a wide variety of fields nowadays. 
	However, there also came out some problems such as the low accuracy of document segmentation, the vagueness of processing results, the bad quality of scanned images, etc. 
	In this context we proposed a document processing system based on Semantic Segmentation, which has a great MIoU rank 97.26 and a perfect document processing functionality. 
	The system use OCRNet to segment documents out meanwhile we make use of Ohem cross entropy loss function to optimize the scanned results. 
	And after that we have a post processing algorithm combined with affine transformation according to the segmentate results to implementate such a outstanding document scanner system.
		
	\end{abstract}
	
	\begin{IEEEkeywords}
		
		Document Scanner, Document Processing, Semantic Segmentation.
		
	\end{IEEEkeywords}
	
	\IEEEpeerreviewmaketitle
	
	\section{Introduction}
	
	% introduce background

	% bring out problems

	% introduce OCRNet

	% take a statement why use it

	% show its MIoU

	% introduce post processing section
	
	\section{Methods}
		\subsection{System Design}

		% show flow charts

		% give the statements.

		\subsection{System Implementation}



			\subsubsection{Image Segmentation}

			% introduce the details of OCRNet

			% compare it with U-Net
			
			% Show the constructure of OCRNet

			% state the loss: OhemCrossEntropyLoss and DiceLoss

			\subsubsection{Document Processing}

			% fitting the keyt points

			% affine trasfromation

		\subsection{Post Processing}

		% why use it : make image clear & keep colorful (maybe need to mention the low space usage).

		% recognize background color of raw scanned images

		% separate foreground color out according to various threshold values of background colors

		% choose several representive color from foreground colors as index colors

	\section{Experiments}

	% the experiment software & hardware environment

	% the used dataset

	% the algorithm comparition between OCRNet and UNet. Attached some pictures.

	% the post processing comparition between ours and adaptive Binarization thresold.
	
	%%%%%%%%%%%%%%%%%%%%%%%%%%%%%%%%%%%%%%%%%%%%%%%%%%%%%%%%%%%%%%%%
	% \cite{panoptic_segmentation} was first proposed by The team of He Kaiming in 2019.
		
	% \begin{figure}[!h]
	% 	\centering
	% 	\includegraphics[width=2.3in]{system_degisn.jpg}
	% 	\caption{System Design}
	% \end{figure}
	
	% \begin{table}[!h]
	% 	\renewcommand{\arraystretch}{1.3}
	% 	\caption{The Comparison Of Adding The Tracker Before And After}
	% 	\centering
	% 	\begin{tabular}{ccccc} \\
	% 		\hline
	% 		Class & Without Tracker & Tracker & True Count & Accuracy\\
	% 		\hline
	% 		car & 5412 & 141 & 125 & 88.65\% \\
	% 		\hline
	% 		bus & 3127 & 4 & 3 & 75.00\% \\
	% 		\hline
	% 		truck & 2743 & 3 & 3 & 100.00\% \\
	% 		\hline
	% 		traffic light & 4151 & 10 & 8 & 80.00\% \\
	% 		\hline
	% 		person & 6189 & 92 & 71 & 77.00\% \\
	% 		\hline
	% 		bicycle & 1433 & 32 & 26 & 81.25\% \\
	% 		\hline
	% 		motorcycle & 3149 & 56 & 52 & 92.85\% \\
	% 		\hline
	% 	\end{tabular}
		
	% \end{table}
	%%%%%%%%%%%%%%%%%%%%%%%%%%%%%%%%%%%%%%%%%%%%%%%%%%%%%%%%%%%%%%%%

	
	\section{Summmary And Conclusion}


	
	% use section* for acknowledgement
	\section*{Acknowledgment}
	
	%This essay is supported by Project to Improve the scientific research basic ability of Middle-aged and Young Teachers (No.2019ky0222,2020KY05033),the Open Funds Guangxi Key Laboratory of Image and Graphic Intelligent Processing, No.GIIP2004, Student's Platform for Innovation and Entrepreneurship Training Program under Grant (No.202010595053. 202010595022)
	
	
	
	% trigger a \newpage just before the given reference
	% number - used to balance the columns on the last page
	% adjust value as needed - may need to be readjusted if
	% the document is modified later
	%\IEEEtriggeratref{8}
	% The "triggered" command can be changed if desired:
	%\IEEEtriggercmd{\enlargethispage{-5in}}
	
	% references section
	
	% can use a bibliography generated by BibTeX as a .bbl file
	% BibTeX documentation can be easily obtained at:
	% http://www.ctan.org/tex-archive/biblio/bibtex/contrib/doc/
	% The IEEEtran BibTeX style support page is at:
	% http://www.michaelshell.org/tex/ieeetran/bibtex/
	%\bibliographystyle{IEEEtran}
	% argument is your BibTeX string definitions and bibliography database(s)
	%\bibliography{IEEEabrv,../bib/paper}
	%
	% <OR> manually copy in the resultant .bbl file
	% set second argument of \begin to the number of references
	% (used to reserve space for the reference number labels box)
	
	\begin{thebibliography}{1}
		
		\bibitem{gps}
		W. Da-chuan, "Algorithm of road information acquirement based on GPS and electronic map," 2010 3rd International Congress on Image and Signal Processing, Yantai, 2010, pp. 4174-4178, doi: 10.1109/CISP.2010.5646704.
		
	\end{thebibliography}
\end{document}

























