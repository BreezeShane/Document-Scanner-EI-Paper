\documentclass[10pt, conference, compsocconf]{IEEEtran}
\usepackage{cite}
\usepackage{subfigure}
\ifCLASSINFOpdf
\usepackage[pdftex]{graphicx}
\graphicspath{{pdf/}{figures/}}
% and their extensions so you won’t have to specify these with
% every instance of \includegraphics
\DeclareGraphicsExtensions{.pdf,.jpeg,.png}
\else
% or other class option (dvipsone, dvipdf, if not using dvips). graphicx
% will default to the driver specified in the system graphics.cfg if no
% driver is specified.
\usepackage[dvips]{graphicx}
% declare the path(s) where your graphic files are
\graphicspath{{…/eps/}}
% and their extensions so you won’t have to specify fthese with
% every instance of \includegraphics
\DeclareGraphicsExtensions{.eps}
\fi

\usepackage{array}
\usepackage{multirow}
\usepackage{longtable}
\usepackage{rotating}
\usepackage[cmex10]{amsmath}
\usepackage{algorithmic}
\usepackage{array}
\usepackage{mdwmath}
\usepackage{mdwtab}
\usepackage{stfloats}
\usepackage{url}
\usepackage[colorlinks,
linkcolor=blue,
anchorcolor=blue,
citecolor=blue]{hyperref}
\hyphenation{op-tical net-works semi-conduc-tor}


\begin{document}
	%
	% paper title
	% can use linebreaks \\ within to get better formatting as desired
	\title{Document Scanner Based on PaddlePaddle}
	
	\author{\IEEEauthorblockN{Chang Lu$^1$\IEEEauthorrefmark{0},
			Xiaochun Lei$^{* 1,2}$\IEEEauthorrefmark{0},
			Junlin Xie$^1$\IEEEauthorrefmark{0}, 
			Xiaolong Wang$^1$\IEEEauthorrefmark{0} and
			XiangBoge Mu$^1$\IEEEauthorrefmark{0}}
		\IEEEauthorblockA{\IEEEauthorrefmark{0}1.School of computer and information security,\\
			Guilin University of Electronic Technology,
			Guilin 541004, China;}
		\IEEEauthorblockA{\IEEEauthorrefmark{0}2.Guangxi Key Laboratory of Image and Graphic Intelligent Processing, \\Guilin 541004, China;}
		\IEEEauthorblockA{\IEEEauthorrefmark{1}Corresponding author's e-mail: glleixiaochun@qq.com}}
	% use for sepcial paper notices
	%\IEEEspecialpapernotice{(Invited Paper)}
	% make the title area
	\maketitle
	\begin{abstract}
		
	Document scanning aims at trasferring the documents in captured photographs into scanned document files. 
	And with the need of convience to gain contents in documents increasing, it is becoming applied to a wide variety of fields nowadays. 
	However, there also came out some problems such as the low accuracy of document positioning, the vagueness of processing results, the bad quality of scanned images, etc. 
	In this paper we proposed a document processing system based on Semantic Segmentation, which has a great MIoU rank 97.26 and a perfect document processing functionality. 
	The system use OCRNet to segment documents out meanwhile we make use of Ohem cross entropy loss function to optimize the scanned results. 
	And after that, we use affine transformation and other post-processing algorithms according to the segmentation results to gain the well-scanned documents.
	\end{abstract}
	
	\begin{IEEEkeywords}
		
		Document Scanner, Document Processing, Semantic Segmentation.
		
	\end{IEEEkeywords}
	
	\IEEEpeerreviewmaketitle
	
	\section{Introduction}
	
	% introduce background

	Document scanning systems have been mature and widely used in amounts of fields such as official business, administration and so on. 
	With the demand of on-device document scanning increasing, systems for document scanning are proposed in order to provide office crowd with convience. 

	% bring out problems

	Originally, for most of the systems like hough transformation!!!, key point regression!!!, the accuracy of document positioning, the legibility of document scanning and the quality of scanned images all cannot satisfy people's needs. 

	% introduce OCRNet

	OCRNet, a object-contextual representation system based on semantic segmentation, is used to be the major component of the system we proposed. 

	% take a statement why use it

	The reasons for choosing it are: 
	
	It solved the context aggregation problem in semantic segmentation by making use of target regional representations to enhance its pixel representations, which evidently improves the quality of segmentation results. 
	The OCR method it mainly uses is light-weighted, effective, rapid and easy to train.  

	% show its MIoU

	And it reached xxx MIoU score based on xxxx dataset. 

	% introduce post processing section

	After we gain the result from OCRNet, we will have it post processed. 
	First, we recognize background color of raw scanned images in it. 
	Then, we separate foreground color out according to various threshold values of background colors. 
	After that, we choose several representive color from foreground colors as index colors. 
	Last, we change the colors in origin image by index colors. 
	
	\section{Methods}
		\subsection{System Design}

		% show flow charts

		% give the statements.

		\subsection{System Implementation}

			\subsubsection{Image Segmentation}

			% introduce the details of OCRNet

			Yuhui Yuan et al proposed OCRNet, a Object-Contextual Representations network for semantic segmentation. 
			It use the construction made up of HRNet, OCR and SegFix. 
			Thereinto, OCR method is divided to three stages. 
			First, it obtains feature representations from backbone network and then estimates a rough semantic segmentation. 
			The result will be the input of OCR method and named soft object regions.
			Second, it computes $k$ vector groups according to soft object regions and feature representations and then make it as object region representations. 
			Third, it computes the relational matrix of pixel representations from the network's bottommost layer and object region representations. 
			Then it computes the weighted summation of the object region features based on the value of each pixel and the object region feature represented in the relationship matrix. 

			% compare it with U-Net
			
			% Show the constructure of OCRNet

			% state the loss: OhemCrossEntropyLoss and DiceLoss

			\subsubsection{Document Processing}

			% fitting the keyt points

			% affine trasfromation

		\subsection{Post Processing}

		% why use it : make image clear & keep colorful (maybe need to mention the low space usage).

		% recognize background color of raw scanned images

		% separate foreground color out according to various threshold values of background colors

		% choose several representive color from foreground colors as index colors

	\section{Experiments}

	% the experiment software & hardware environment

	% the used dataset

	% the algorithm comparition between OCRNet and UNet. Attached some pictures.

	% the post processing comparition between ours and adaptive Binarization thresold.
	
	%%%%%%%%%%%%%%%%%%%%%%%%%%%%%%%%%%%%%%%%%%%%%%%%%%%%%%%%%%%%%%%%%%%%%%%%%%%%%%%%%%%%%%%%%%%%%%%%%%%%%%%%%%%%%%%%%%%%%%%%%%%%%%%%

	% \cite{panoptic_segmentation} was first proposed by The team of He Kaiming in 2019.
		
	% \begin{figure}[!h]
	% 	\centering
	% 	\includegraphics[width=2.3in]{system_degisn.jpg}
	% 	\caption{System Design}
	% \end{figure}
	
	% \begin{table}[!h]
	% 	\renewcommand{\arraystretch}{1.3}
	% 	\caption{The Comparison Of Adding The Tracker Before And After}
	% 	\centering
	% 	\begin{tabular}{ccccc} \\
	% 		\hline
	% 		Class & Without Tracker & Tracker & True Count & Accuracy\\
	% 		\hline
	% 		car & 5412 & 141 & 125 & 88.65\% \\
	% 		\hline
	% 		bus & 3127 & 4 & 3 & 75.00\% \\
	% 		\hline
	% 		truck & 2743 & 3 & 3 & 100.00\% \\
	% 		\hline
	% 		traffic light & 4151 & 10 & 8 & 80.00\% \\
	% 		\hline
	% 		person & 6189 & 92 & 71 & 77.00\% \\
	% 		\hline
	% 		bicycle & 1433 & 32 & 26 & 81.25\% \\
	% 		\hline
	% 		motorcycle & 3149 & 56 & 52 & 92.85\% \\
	% 		\hline
	% 	\end{tabular}
		
	% \end{table}

	%%%%%%%%%%%%%%%%%%%%%%%%%%%%%%%%%%%%%%%%%%%%%%%%%%%%%%%%%%%%%%%%%%%%%%%%%%%%%%%%%%%%%%%%%%%%%%%%%%%%%%%%%%%%%%%%%%%%%%%%%%%%%%%%
	
	\section{Summmary And Conclusion}


	
	% use section* for acknowledgement
	\section*{Acknowledgment}
	
	%This essay is supported by Project to Improve the scientific research basic ability of Middle-aged and Young Teachers (No.2019ky0222,2020KY05033),the Open Funds Guangxi Key Laboratory of Image and Graphic Intelligent Processing, No.GIIP2004, Student's Platform for Innovation and Entrepreneurship Training Program under Grant (No.202010595053. 202010595022)
	
	
	
	% trigger a \newpage just before the given reference
	% number - used to balance the columns on the last page
	% adjust value as needed - may need to be readjusted if
	% the document is modified later
	%\IEEEtriggeratref{8}
	% The "triggered" command can be changed if desired:
	%\IEEEtriggercmd{\enlargethispage{-5in}}
	
	% references section
	
	% can use a bibliography generated by BibTeX as a .bbl file
	% BibTeX documentation can be easily obtained at:
	% http://www.ctan.org/tex-archive/biblio/bibtex/contrib/doc/
	% The IEEEtran BibTeX style support page is at:
	% http://www.michaelshell.org/tex/ieeetran/bibtex/
	%\bibliographystyle{IEEEtran}
	% argument is your BibTeX string definitions and bibliography database(s)
	%\bibliography{IEEEabrv,../bib/paper}
	%
	% <OR> manually copy in the resultant .bbl file
	% set second argument of \begin to the number of references
	% (used to reserve space for the reference number labels box)
	
	\begin{thebibliography}{1}
		
		\bibitem{gps}
		W. Da-chuan, "Algorithm of road information acquirement based on GPS and electronic map," 2010 3rd International Congress on Image and Signal Processing, Yantai, 2010, pp. 4174-4178, doi: 10.1109/CISP.2010.5646704.
		% Vyacheslavovich T D, Mikhailovich E A, Petrovich N D. Advanced Hough-based method for on-device document localization[J]. Компьютерная оптика, 2021, 45(5): 702-712.
		% Advanced Hough-based method for on-device document localization

		% Segmentation Transformer: Object-Contextual Representations for Semantic Segmentation

		% Advanced Hough-based method for on-device document localization

		% Real-Time Document Localization in Natural Images by Recursive Application of a CNN
		
	\end{thebibliography}
\end{document}
